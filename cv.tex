\documentclass[11pt,a4paper]{moderncv}
\moderncvtheme[red]{classic}                
\usepackage[utf8]{inputenc}
\usepackage[top=0.7cm, bottom=0.7cm, left=1cm, right=1cm]{geometry}
% Largeur de la colonne pour les dates
\setlength{\hintscolumnwidth}{3.5cm}
\firstname{Jean-Patrick}
\familyname{Debaëne}
\title{Développeur backend Python/Django}              
\address{68 rue Winston Churchill}{59160 Lomme}    
\email{jp.debaene@gmail.com}                      
\mobile{06 63 39 80 26} 
\extrainfo{31 ans -- Pacsé -- 2 enfants}
\photo[80pt][0pt]{photo.jpg}
\homepage{http://github.com/jeeppy}
\begin{document}
\maketitle

\section{Expériences professionnelles}
\cventry{Février 2015 \\à aujourd'hui}{Développeur backend}{CIM}{Villeneuve d'Ascq}{France}{
Développement de solutions en mode SAAS destinées à l'assurance de personnes (tarificateur, produit designer, gestion de pièces justificatives, DSN) au sein d'une équipe de 15 personnes:
\begin{itemize}
\item Développement Python avec le frameworks Django
\item Développement d'API Rest avec Django Rest Framework
\item Développement de scripts shell (déploiement et exports de données)
\item Tests unitaires
\item Contrôle de version, intégration continue, revue de code, scrum meeting
\item Base de données PostgreSQL
\end{itemize}}
\cventry{Avril 2010 \\à février 2015}{Concepteur développeur}{CIM}{Villeneuve d'Ascq}{France}{
Développement d'un progiciel dédié aux mutuelles et assurances de personnes. En particulier, des modules de cotisations et des échanges de données entre les différents acteurs du marché : 
\begin{itemize}
\item Développement Powerbuilder
\item Développement de requêtes SQL
\item Base de données DB2
\end{itemize}}
\cventry{Septembre 2008 \\à octobre 2009}{Développeur Windev}{GID}{Douai}{France}{
Développement de diverses solutions sous Windev :
\begin{itemize}
\item Solution d'administration réseau
\item Gestion de dossier d'un cabinet d'architecture
\item Gestion de biens pour une agence immobilière
\end{itemize}}
\cventry{Juin 2007 \\à Août 2007}{Développeur C++}{Sofintec}{Amiens}{France}{
Développement en C++ et QT d'un logiciel éducatif à destination des enfants de primaire.}
\cventry{Janvier 2016 \\à février 2016}{Développeur web}{Auto Picardie}{Amiens}{France}{
Développement d'un site vitrine permettant la mise en vente de véhicules d'occasions.}
\cventry{Juin 2015}{Technicien informatique}{Centre hospitalier}{Beauvais}{France}{}
\section{Formations}
\cventry{2016}{Pyconfr}{Rennes}{}{}{Conférence autour du Python organisée par AFPy}
\cventry{2009}{Chargé de projets en systèmes informatiques appliqués}{Efficom Lille}{}{}{}
\cventry{2007}{Diplôme universitaire de technologie d'informatique}{IUT d'Amiens}{}{}{Spécialité génie logiciel}
\cventry{2006}{Brevet de technicien supérieur d'informatique}{Lycée St Rémi d'Amiens}{}{}{Spécialité développeur d'application}
\cventry{2004}{Baccalauréat}{Lycée Félix Faure de Beauvais}{}{}{Série STT Informatique et Gestion}

\section{Compétences}
\cvitem{Langages}{Python, HTML, Javascript, SQL, Powerbuilder, Shell}
\cvitem{Frameworks}{Django, Django Rest Framework, \textit{VueJS, AngularJS}}
\cvitem{Base de données}{PostgreSQL, DB2}
\cvitem{Outils}{Git, Centry}
\cvitem{Systèmes}{Windows, Linux (Ubuntu, Centos)}
\cvitem{Méthodes}{Agile, Scrum, Merise}
\cvlanguage{Anglais}{Technique}{}

\section{Projet personnel}
\cventry{2017}{Carnet d’entraînement}{}{}{}{
Développement d'un carnet d'entraînement, récapitulatif des entraînements avec des statistiques. Les entraînements sont synchronisés depuis l'API de Strava et distribués au front via des APIs REST :
\begin{itemize}
\item Développement backend en Python/Django
\item Développement d'Api REST avec Django REST Framework
\item Développement frontend avec HTML, CSS et Javascript (framework VueJS)
\item Contrôle de version : Github (\url{http://github.com/jeeppy})
\item Base de données : PostgreSQL
\end{itemize}}

\section{Centres d'intérêt}
\cvitem{Sport}{Triathlon, football}
\cvitem{Jeux de société}{Bang, Mysterium, Pandémie, Désert interdit, Noé, Smash Up...}
\cvitem{Cinéma}{Films et séries US}

\end{document}